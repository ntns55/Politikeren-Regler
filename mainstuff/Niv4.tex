\chapter*{Niveau 4}
\addcontentsline{toc}{chapter}{Niveau 4}

Dine ord er som torden over bjergene, dine løfter skrevet i blod og skæbne. Hvor du går, bøjer virkeligheden sig. Riger rejses med dine tanker, guder styrtes af dine intriger. Du har bøjet historiens strøm under din tunge, og selv kosmiske kræfter lytter bag kulisserne.

\begin{table}[H]
    \centering
    \begin{tabular}{|p{0.50\textwidth}|p{0.25\textwidth}|}
    \rowcolor{cerulean!80}\hline
        Evne navn & Pris i XP \\\hline
         Akademiske Kontakter Niv. 4 &4 \\\hline
         Diplomatisk magt Niv. 2 & 3 \\\hline
         Glemte kontakter & 4 \\\hline
         Guld for Magt & 2 \\\hline
         Kirkeskat Niv. 2 & 2 \\\hline
         Kirkeskat Niv. 3 & 4 \\\hline
         Social Kamæleon & 2 \\\hline
         Triage & 3 \\\hline
    \end{tabular}
\end{table}
\section*{Evne beskrivelse}
\addcontentsline{toc}{section}{Evne beskrivelse}

\subsection*{Akademiske Kontakter Niv. 4}
\addcontentsline{toc}{subsection}{Akademiske Kontakter Niv. 4}

Når du køber denne evne første gang kan du vælge 1 af dine stier fra et foregående niveau som vil blive hævet til niveau 4.\\
Du kan købe denne sti flere gange, hver gang vælge 1 sti som vil blive hævet til niveau 4, såfremt den allerede er niveau 3. \\

\textbf{Brug af evnen:}\\
Hvert niveau af Akademiske Kontakter giver dig én tjeneste pr. spilgang, svarende til niveauet. Tjenesterne er midlertidige og gælder kun for den pågældende spilgang. Her er nogle eksempler på hvad en tjeneste fra den relevante sti kan bruges til i niveau 2:\\
\begin{itemize}
    \item Brutalitet: Få udleveret en rune genstand midlertidigt, eller send hvad der svarer til Inkvisationen for din faktion efter et mål.
    \item Diplomati: Du har opnået indflydelse blandt den politiske elite, muligvis kongefamilien, ministerrådet, eller en rådgivningsgruppe af adelige. Her kan du bede om tjenester som statsstøtte til projekter, større fordele i diplomatiske forhandlinger, indflydelse på nationale beslutninger eller alliancer, eller adgang til dybt fortrolige oplysninger, som kun de mest betroede kender til. Du kan også få effekten fra et overtaget "Vigtigt Sted" hvis du har overtaget over halvdelen af alle områder, i stedet for at skulle have overtaget alle områder.
    \item Kirken: Du har nu adgang til kirkens øverste ledelse, f.eks. patriarker, ærkebiskopper eller orakler. På dette niveau kan du anmode om tjenester såsom indflydelse på kirkens officielle doktriner, hemmelige råd fra kirkens orakler, eller opnåelse af fordelagtige alliancer på tværs af trosretninger. Du kan også modtage en specifik midlertidig skriftrulle som en præst kan kaste.
    \item Luskeri: Denne spilgang må du vælge at agere for en anden faktion på faktionskortet, eller du kan forfalske et dokument en anden spiller har efterspurgt at få udleveret.
    \item Viden: Du kan få udlevert en magisk genstand midlertidigt eller en specifik midlertidig skriftrulle. 
\end{itemize}

\subsection*{Diplomatisk magt Niv. 2}
\addcontentsline{toc}{subsection}{Diplomatisk magt Niv. 2}
Det koster dig 50\% mindre Fjend at udføre en handling på faktionskortet. Du skal tydeligt promovere faktionen du repræsentere uden for spillet. Denne kan ikke bruges sammen med effekten fra Niveau 1.

\subsection*{Glemte kontakter}
\addcontentsline{toc}{subsection}{Glemte kontakter}
Du får en kontakt i det glemte akademi. Denne evne kan hjælpe dig med at få hemmelig eller obskur viden ellers ikke tilgængeligt.

\subsection*{Guld for Magt}
\addcontentsline{toc}{subsection}{Guld for Magt}
Du kan bruge kontakter flere gange ved at betale for dem. Du skal have adgang til kontakten på normal hvis, men du kan bruge evnen igen ved at betale kontaktens ønskede niveau gange 10 Fjend. Det vil sige at en niveau 3 kontakt koster 30 Fjend at bruge igen.

\input{../Evne-Ordbog/Kirkeskat/Kirkeskat Niv 2.tex}

\input{../Evne-Ordbog/Kirkeskat/Kirkeskat Niv 3.tex}

\subsection*{Social Kamæleon}
\addcontentsline{toc}{subsection}{Social Kamæleon}
Du må vælge en faktion mere som du repræsenterer både med hensyn til Akademiske Kontakter, men også på faktionskortet. 

\input{../Evne-Ordbog/Triage.tex}