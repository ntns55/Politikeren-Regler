\chapter*{Niveau 2}
\addcontentsline{toc}{chapter}{Niveau 2}
At samle en gruppe af folk under dit banner kræver afsindige mængder tid, penge, og charisma. Du har gjort dit for at samle din flok og du begynder at være en politisk spiller som selv de mellemstore noble er opmærksomme på. Husk at den hurtigste vej til en tidlig død er at stole på dine venner.

\begin{table}[H]
    \centering
    \begin{tabular}{|p{0.50\textwidth}|p{0.25\textwidth}|}
    \rowcolor{cerulean!80}\hline
        Evne navn & Pris i XP \\\hline         
         Akademiske Kontakter Niv. 2 & 2 \\\hline
         Gruppestifter Niv. 2 & 1 \\\hline
         Helbred Sår & 2 \\\hline
         Investorer Niv. 2 & 2 \\\hline
         Lommetyveri Niv. 1 & 2 \\\hline
         Påfør gift & 2 \\\hline
         Reparer Rustning Niv. 1 & 2 \\\hline
    \end{tabular}
\end{table}
\section*{Evne beskrivelse}
\addcontentsline{toc}{section}{Evne beskrivelse}

\subsection*{Akademiske Kontakter Niv. 2}
\addcontentsline{toc}{subsection}{Akademiske Kontakter Niv. 2}

Når du køber denne evne første gang kan du vælge 2 af dine stier fra et foregående niveau som vil blive hævet til niveau 2.\\
Du kan købe denne sti flere gange, efter første gang kan du vælge 1 sti som vil blive hævet til niveau 2, såfremt den allerede er niveau 1. \\

\textbf{Brug af evnen:}\\
Hvert niveau af Akademiske Kontakter giver dig én tjeneste pr. spilgang, svarende til niveauet. Det vil sige at du kan bruge både en niveau 1 tjeneste og en niveau 2 tjeneste fra samme sti. Tjenesterne er midlertidige og gælder kun for den pågældende spilgang. Her er nogle eksempler på hvad en tjeneste fra den relevante sti kan bruges til i niveau 2:\\
\begin{itemize}
    \item Brutalitet: Du kan indsende kontrakter på andres liv, som dine lysky kontakter vil prøve at udføre.
    \item Diplomati: Du har nu indflydelse på et regionalt niveau og kan få adgang til personer såsom grever, baroner eller distriktsguvernører. Du kan få lov til at udføre et ryk med din valgte faktion på faktionskortet, uden at betale for dette. Du kan også repræsentere faktionen når du skriver breve med videre.
    \item Kirken: Du har nu indflydelse over biskopper eller tilsvarende ledere på regionalt niveau, hvilket giver dig mulighed for at få større tjenester, såsom godkendelse til religiøse ceremonier, opnåelse af ressourcer eller donationer fra menigheder, og eller mellem store velsignelser så som helligt våben til alle af dine følgere.
    \item Luskeri: Få skabt et stykke propaganda om en person eller gruppe du ønsker at nedgøre. 
    \item Viden: Du kan efterspørge alt viden, en midlertidig skriftrulle, eller få et tilfældigt stykke viden udleveret. 
\end{itemize}

\subsection*{Gruppestifter Niv. 2}
\addcontentsline{toc}{subsection}{Gruppestifter Niv. 2}

Faktionsbonusser:\\
Professionsspecifikke Buffs: Når faktionens medlemmer befinder sig på faktionens territorie, modtager de professionsspecifikke bonusser som styrker deres evner. Disse bonusser gives afhængigt af professionen:
\begin{itemize}
    \item Alkymist: Kan på deres territorie brygge en “Nervegift” eller “Naturens Bryg” per medlem til start i scenariet. Disse kan kun drikkes af den spiller den er brygget til.
    \item Skovfoged: Skovfoged kan bruge deres territorie som helliggrund.
    \item Præst: Præster kan med “gudens velsignelse” vælge en ekstra forkæmper. Deres forkæmper evne er kun aktiv på territoriet.
    \item Handelsmand: Handelsmanden får dobbelt effekt fra deres "Ambassadør" evne, og må behandle politikeren som en handelspartner for evnen "Handelspartner"
    \item Tyv: Tyve på territoriet får evnen til én gang per spilgang/Hvert lommetyveri at udføre et perfekt lommetyveri, som ikke kan opdages. Derudover kan tyven på territoriet vælge at forsvare deres faktionskiste, hvilket giver dem mulighed for at fordoble deres NK (Nævekamp) mod enhver, der forsøger at bryde ind i kisten.
\end{itemize}
Faktionskiste:\\
Diplomaten får tilladelse til at medbringe en faktionskiste på territoriet, som kan sikres med en niveau 2 lås. Denne lås er gratis og kan kun anvendes på kisten.\\
Placering af Faktionskiste: Kisten skal stå synligt på territoriet og må ikke skjules i terrænet (såsom under blade, i træer, eller bag strukturer). Kisten er en symbolsk skat for faktionen og fungerer som centrum for deres territorie.


\input{../Evne-Ordbog/Helbred sår.tex}

\input{../Evne-Ordbog/Investorer/Investorer Niv 2.tex}

\subsection*{Lommetyveri Niv. 1}
\addcontentsline{toc}{subsection}{Lommetyveri Niv. 1}
Med Lommetyveri Niv. 1 kan du lægge din hånd på offerets pung i 15 sekunder, for derefter at stjæle alt (undtagen de sidste 3 mønter) i pungen. Offeret ved ikke, at de er blevet bestjålet. \\
Dette virker dog ikke, hvis offeret siger "6 sans, immun" efter tyven har bekendtgjort lommetyveriet efter de 15 sekunder. Grundet den generelle evne 6 sans Niv. 1 gør en person immun overfor Lommetyveri Niv. 1 eller, hvis offeret har en lås på sin pung.

\subsection*{Påfør gift}\addcontentsline{toc}{subsection}{Påfør gift}
Du kan påføre gifte, som vil give skade på et våben. Våbnet skal have et blad, så som kniv, sværd eller ligende, og må ikke være et projektil, såsom pile eller en kastet kastekniv. Giften vil kun vare på det næste slag eller til våbnet gives til en anden person, hvor efter giften vil forsvinde.\\
Når et våben bruges på denne måde skal ordene: "Gift kniv x i skade" siges. Her vil x være hvor meget skade du giver med giften. Skulle giften have andre effekter end skade skal disse ikke siges da de ignoreres.

\input{../Evne-Ordbog/Reparere rustning Niv. 1.tex}